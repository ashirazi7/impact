%sample file for Modelica 2015 Conference paper

\documentclass[11pt,a4paper,twocolumn]{article}
\usepackage{graphicx}
% uncomment according to your operating system:
% ------------------------------------------------
%\usepackage[latin1]{inputenc}    %% european characters can be used (Windows, old Linux)
\usepackage[utf8]{inputenc}     %% european characters can be used (Linux)
%\usepackage[applemac]{inputenc} %% european characters can be used (Mac OS)
% ------------------------------------------------
\usepackage[T1]{fontenc}     %% get hyphenation and accented letters right
\usepackage{mathptmx}        %% use fitting times fonts also in formulas
\usepackage{amsmath,amssymb} %% Nice maths
\usepackage[round]{natbib}   %% author-year style referencing
\usepackage{doi}             %% Create cor­rect hy­per­links for DOI num­bers
\usepackage{booktabs}        %% Nice tables
\usepackage{hyperref}
\usepackage{color}
\usepackage[labelfont=bf, labelsep=period, font=small]{caption}  %% Get bold Figure/Table caption
               %% Set separator in figures to '.', set fontsize to small
\usepackage{authblk}         %% Prepare author and affiliation blocks
\usepackage{courier}         %% For proper courier fonts in texttt
\usepackage{listings}        %% For code sections
%\usepackage[bw]{dtsyntax}    %% For Modelica code

% do not change these lines:
\pagestyle{empty}                %% no page numbers!
\usepackage{geometry}            %% please don't change geometry settings!
\geometry{left=20mm, right=20mm, top=25mm, bottom=25mm, noheadfoot}

\hypersetup{
  pdftitle = {Where impact got going},
  pdfauthor = {Michael Tiller and Dietmar Winkler},
  pdfsubject = {11th International Modelica Conference 2015},
  pdfkeywords = {Modelica, package manager, GitHub, dependency management, golang},
  hidelinks,
  pdfpagelayout=SinglePage}

\renewcommand{\normalsize}{\fontsize{10.5pt}{12.3pt}\selectfont}
\renewcommand{\small}{\fontsize{9.5pt}{11.1pt}\selectfont}
\renewcommand{\footnotesize}{\fontsize{8.5pt}{9.9pt}\selectfont}

%% Modelica code configuration
% \lstset{language = Modelica,
%        basicstyle=\fontsize{9pt}{10.5pt}\selectfont,
%        backgroundcolor = \color{white}}

% usefull commands
\newcommand{\myr}{\textsuperscript{\textregistered}}
\newcommand{\ud}{\mathrm{d}}
\newcommand{\matx}[1]{\mathbf{#1}}
\newcommand{\impact}{\texttt{impact}} % impact is going to get used quite a lot :)
\newcommand{\code}[1]{\texttt{#1}} % make quoting code text a bit simpler


% begin the document
\begin{document}
\thispagestyle{empty}

\title{\textbf{Where \impact\ got \emph{go}ing}}
\renewcommand\Authfont{\large}        %% Set author font
\renewcommand\Affilfont{\normalsize}       %% Set affiliation font
\renewcommand\Authsep{\quad}                     %% Set text between authors names
\renewcommand\Authand{\quad}                     %% Set text between authors names
\renewcommand\Authands{\quad}                    %% Set text between authors names
\author[1]{Michal Tiller}
\author[2]{Dietmar Winkler}
\affil[1]{\href{http://xogeny.com}{Xogeny Inc.}, USA, {\small \href{mailto:michael.tiller@xogeny.com}{\nolinkurl{michael.tiller@xogeny.com}}}}
\affil[2]{\href{http://www.hit.no}{Telemark University College}, Norway, {\small\href{mailto:dietmar.winkler@hit.no}{\nolinkurl{dietmar.winkler@hit.no}}}}
% \title{\textbf{Int. Modelica Conf. 2015 Paper Title}}
% \author{{\large
% Author Name$^1$ \quad Author Name$^1$ \quad Author Name$^2$\vspace{2mm}\\
%   {}$^1$Department, University, Country, \textsf{\{name1,name2\}@university.org}\\
%   {}$^2$Company, Contry, \textsf{name3@company}}


\date{} % <--- leave date empty
\maketitle\thispagestyle{empty} %% <-- you need this for the first page
\abstract{
TODO
}

\noindent\emph{Keywords: Modelica, package manager, GitHub, dependency management, golang}

\section{Introduction}

\subsection{Motivation}

* encourage best practices (version control)

* make it easy for library developers to publish work

* make it easy for people to locate and use existing libraries

* **healthy ecosystem**

\subsection{History}

* Was, at one point, a single file Python script

* Was Python, switched to Golang for reasons that will be discussed.

\section{Requirements}

After initial version, we thought about what works and what doesn't.

* Static binary (no runtime, easy install)
  - Discuss issues with Python (https, 2 vs 3, runtime)

* easy cross-compilation

* reasonably fast (important for dependency resolution)

\section{Version Numbering}

* Semantic versioning

* Best practices

* Normalization
  - Used for Modelica

\section{Indexing}

\subsection{Sources}

* tags

\subsection{Repository Structure}
\subsubsection{Conventions}
\subsubsection{impact.json}

\subsection{Handling Forks}

* Present basic data structures assuming uniqueness

* forks (URIs, disambiguation)

\subsection{Schema}

* JSON Schema?

\section{Installation}

Talk about what happens during installation.

\subsection{Dependency Resolution}

* Basic algorithm (show some graphs)

* Extracting explicit dependency information from Modelica

* Implicit dependency information from semantic versions

* Backtracking and why it isn't so bad (give some performance numbers here)

\subsection{Directory Structure}

* No version numbers

* In a version control context

\subsection{impact.proj}

* --save option?

* .gitignore and repopulation

* disambiguation

\section{Go Implementation}

* package structure

* 3rd party libraries
  - GitHub API
  - Semver

* GitHub API tokens

* CI

\section{Additional Features}

\subsection{Searching}

* Ordering (matching URI/source then by stars)  

\section{Future Development}

* Bitbucket

* Subversion

* Intranet applications

* Web based search

\section{Conclusion}

* Different files (impact.json, impact.proj, impactrc and their purposes)

%%% Remove the following line once we got real citations in place.
\nocite{*}

%--------------------------------------------------------------------------------
% References using bibtex
\small
\bibliographystyle{plainnat}
\bibliography{impact}
\normalsize

\end{document}
